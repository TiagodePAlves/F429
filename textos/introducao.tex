Em sistemas de controle, é de suma importância monitorar como certas grandezas físicas se comportam com o passar do tempo. Para tal observação, analisar tanto a derivada quanto a integral dessas grandezas mostra-se extremamente importante. em eletrônica analógica, a forma mais comum de cumprir esta tarefa é empregando circuitos diferenciadores e integradores.

A forma mais comum de se construir tais circuitos é empregando filtros passivos, mais especificamente, filtros RC e RL. Ao observar circuitos desse tipo em relação ao domínio de tempo. dadas frequências especificas, o sinal de saída desses sistemas é capaz de representar a derivada ou integral do sinal de entrada e assim monitorar o sinal inicial.

Muitos sistemas físicos apresentam comportamento oscilatório. Filtros RLC, os chamados filtros ressonantes, também  exibem o mesmo tipo de comportamento. Dessa forma, observar como os mesmo se comportam durante as oscilações é útil para estudar como esse fenômeno ocorre.

Considerando os sistemas citados, o objetivo deste experimento é construir um circuito capaz de integrar sinais arbitrários com frequências maiores que $\SI{50}{\kilo\hertz}$ e diferenciar sinais com frequência menos que \SI{500}{\hertz}. Além disso, também temos como objetivo estudar o fenômeno transiente utilizando um circuito RLC que possa operar em dois regimes de amortecimento, subamortecido e sobreamortecido, de forma que, no regime estacionário, a tensão de saída se estabilize ao redor da tensão de entrada.
