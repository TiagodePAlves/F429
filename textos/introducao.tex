Dentre todos os fenômenos ondulatórios, a difração possui uma posição de destaque. Através dela, Thomas Young no início do século XIX comprovou a natureza ondulatória da Luz. A partir disso, a teoria corpuscular da luz, concedida por Isaac Newton, deixou de ser considerada. 

Este experimento traz uma reconstrução de parte do experimento de Thomas Young, conhecido por “Experimento da Dupla Fenda”, bem como uma extensão dos resultados para a difração da luz em fendas simples, duplas e múltiplas. 

Além da comprovação da natureza ondulatória da luz é possível determinar o comprimento de onda da luz incidente a partir dos padrões de interferência formados pela rede de difração. 

Assim, este experimento tem como objetivos: observar os efeitos de difração em fendas simples e os efeitos de interferência em fendas simples e múltiplas; verificar as previsões do modelo de difração de Fraunhofer para fendas simples e sua validade para múltiplas fendas, comparando a medida da largura de uma fenda a partir de padrões de difração com a medida realizada com microscópio metrológico.

