A dispersão cromática, observada no século XVII pelo físico e matemático britânico Isaac Newton (1643 - 1727), é um fenômeno de extrema importância, inclusive histórica. Decorre das observações de Newton ideias importantes sobre a Teoria das Cores. 

O experimento relatado tem por objetivo o estudo da dispersão de um prisma de vidro a partir da decomposição da luz utilizando-se lâmpadas de diferentes espectros. Prismas, além de sua importância prática em instrumentos ópticos como periscópios, binóculos e lentes para algumas ametropias da visão, podem ser utilizados como espectrômetros. 

Utilizando-se um espectrômetro, é possível dispersar um feixe de luz em frequências aproximadamente monocromáticas, o que permite um estudo detalhado das propriedades da luz visível. A construção desse espectro possibilita um estudo dirigido das diferentes faixas de ondas eletromagnéticas.
     
É possível obter uma equação que permita determinar o índice de refração de um dado material para diferentes comprimentos de onda que nele são incididas, a partir do desvio mínimo de dispersão de lâmpadas de diferentes constituições. Em suma, este experimento busca essa relação para um prisma de vidro, assim, determinando uma curva que possibilite a utilização do prisma como espectrômetro, também definindo a resolução espectral do mesmo.
