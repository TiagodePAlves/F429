\todo[inline,color=yellow]{Listar equipamentos (modelo e número de identificação quando possível) e componentes (valores nominais) utilizados.}

Para a primeira parte do experimento, foi montado, em uma \textit{protoboard}, um filtro passa-alta na forma de  circuito RC, como representado em \cref{fig:passaalta} . No sistema em questão foi utilizado um capacitor de capacitância nominal \SI{47}{\nano\farad} e um resistor de resistência nominal \SI{470}{\ohm}. Com o circuito devidamente pronto, utilizando o osciloscópio \texttt{Tektronix TBS1000 Series}, juntamente com o gerador de função \texttt{BK Precision 4052}, ajustado para ondas senoidais, foram coletados $100$ pontos igualmente espaçados para um \textit{sweep} de frequência de \SI{10}{\hertz} a \SI{10}{\kilo\hertz} registrando as tensões de saída $V_{out}$ em cada ponto. Em um computador, foi utilizado o \textit{software} \texttt{PyLab}, automatizando as medições. É importante salientar que o \textit{range} de frequências analisado foi escolhido de forma que abrangesse pelo menos uma ordem de grandeza acima e abaixo do que o experimento buscava analisar, permitindo verificar a eficácia do filtro.
\par
Para a segunda parte do experimento, seguindo o método da primeira parte, em uma \textit{protoboard}, foi montado um filtro rejeita-banda na forma de circuito RLC, assim como representado em \ref{fig:rejeitabanda}. Para tal circuito foi utilizado indutor de indutância nominal \SI{48,60}{\milli\henry}, capacitor de \SI{470}{\nano\farad} e resistor de \SI{470}{\ohm}. Em seguida, da mesma forma realizada anteriormente, foram coletados dados de tensão de saída ($V_{out}$) para um \textit{sweep} de frequência de \SI{10}{\hertz} a \SI{10}{\kilo\hertz}. Sendo que o \textit{range} em questão foi empregado de formar a cumprir os objetivos do experimento, tendo a mesma justificativa da seção anterior. Ressalva-se que o fato do \textit{range} de frequências ser o mesmo em ambas as partes é uma coincidência.

Em sequência, os dados anteriormente coletados foram utilizados para a construção de diagramas de Bode para os dois filtros. A seguir, na seção \ref{resultados} os mesmos serão abordados com maior profundidade.

\begin{figure}
    \centering

    \begin{subfigure}[t]{0.3\textwidth}
        \centering
        \begin{circuitikz}[scale=1.2]

  \draw (0, 2)	% linha superior
  node[above] {$V_{in}$}
  to [capacitor, o-, l=$C$] ++(2,0)
  to [short, -o] ++(1,0)
  to ++(0,0) node[above] {$V_{out}$};

  \draw (2,0)		% aterramento
  node[ground] {}
  to [resistor, -, l_=$R$] ++(0,2);

\end{circuitikz}

        \caption{Passa-altas}
        \label{fig:passaalta}
    \end{subfigure}
    \qquad
    \begin{subfigure}[t]{0.3\textwidth}
        \centering
        \begin{circuitikz}[scale=1.2]

  \draw (0, 2)    % linha superior
  node[above] {$V_{in}$}
  to [resistor, o-, l=$R$] ++(2,0)
  to [short, -o] ++(1,0)
  to ++(0,0) node[above] {$V_{out}$};

  \draw (2,0)	    % aterramento
  node[ground] {}
  to [inductor, l_=$L$] ++(0,1)
  to [capacitor, l_=$C$] ++(0,1);

\end{circuitikz}

        \caption{Rejeita-banda}
        \label{fig:rejeitabanda}
    \end{subfigure}

    \caption{Circuitaria.}
    \label{fig:circuitos}
\end{figure}