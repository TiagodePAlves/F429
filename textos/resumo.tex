Neste experimento foram estudados padrões de difração em fendas simples, duplas e múltiplas a fim de verificar a capacidade da utilização do modelo de difração de Fraunhofer na determinação da largura e separação de fendas. A fim de cumprir os objetivos deste experimento, luz aproximadamente monocromática de cor verde foi incidida em uma placa de vidro contendo diversas configurações de fendas e analisando os padrões de difração formados em um anteparo distante da fonte de luz. Ao fim do experimento os resultados obtidos para as características das fendas foram calculados pelo modelo citado e comparados com resultados obtidos utilizando um microscópio metrológico. Nossos resultados mostraram grande compatibilidade do modelo teórico com os valores experimentais, validando a possibilidade de se utilizar estes modelos teóricos para medidas mais complexas.
