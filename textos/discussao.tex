A partir dos resultados observados na seção \ref{resultados}, a proximidade entre os valores obtidos a partir de cálculos, seguindo o modelo de Fraunhofer e aqueles obtidos com microscópio metrológico, é possível observar que o modelo citado demonstrou grande coerência. Os valores obtidos tanto para a largura das fendas quanto para a separação das mesmas foram condizentes para ambas as formas de medição, ou seja, demonstraram resultados similares, verificando as previsões de Fraunhofer.

A tabela \ref{tab:fend_b2} com as dimensões das fendas B também mostra que as medidas calculadas pelo modelo teórico tem bem menos variância nos resultados. Sendo que a espessura deveria ser a mesma para todas as fendas, mudando apenas suas separações, vemos que para os valores calculados, o desvio padrão dos resultados é bem menor que os mesmo valores medidos com o microscópio. Isso mostra que o modelo teórico tem resultados mais confiáveis e também que a incerteza encontrada é condizente com os resultados, se aproximando bem do desvio padrão das 3 medidas nos dois casos.

Em todos os conjuntos de fendas, de A a C, as grandezas mensuradas tiveram diferenças numéricas da ordem de micrômetros. Assim evidenciou-se que, pelo modelo de Fraunhofer é possível determinação da das grandezas das fendas de difração.

Para as fendas múltiplas de mesmas dimensões, a regressão na figura \ref{fig:log} deu os resultados na tabela \ref{tab:regres} com $B \approx -1$. Com isso, o experimento também demostrou que a largura dos máximos primários de interferência depende inversalmente da quantidade de fendas, não apenas da separação delas. Novamente, mais um resultado que seria esperado, seguindo o modelo teórico.

Na medição do fio de cabelo, o resultado foi algo próximo de \SI{60}{\micro\meter} (tab. \ref{tab:difcab}). Esse valor bate com outros resultados de fontes variadas, como também com os valores das fontes da publicação\cite{ref:cabelo} de Brian Ley sobre o assunto. Nessa dissertação a margem mais abrangente é de \SI{17}{\micro\meter} até \SI{181}{\micro\meter}, porém vários dos outros valores apresentados orbitavam em torno do resultado encontrado no experimento. Novamente, mais um método válido para a medição usada.

Em relação às imagens observadas na figura \ref{fig:D}, as mesmas demonstraram que o padrão de difração observado no anteparo não só depende da largura, separação e quantidade de fendas, mas como também do formato e disposição das fendas. Assim, evidenciando que o padrões de difração e interferência observados podem ser muito diversos, dependendo fortemente das condições experimentais presentes, em especial, das características dimensionais das fendas.
