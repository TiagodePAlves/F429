A partir dos resultados observados na seção \ref{resultados}, a proximidade entre os valores obtidos a partir de cálculos, seguindo o modelo de Fraunhofer e aqueles obtidos com microscópio metrológico, é possível observar que o modelo citado demonstrou grande coerência. Os valores obtidos tanto para a largura das fendas quanto para a separação das mesmas foram condizentes para ambas as formas de medição, ou seja, demonstraram resultados similares, verificando as previsões de Fraunhofer.

Para todos os conjuntos de fendas, de A a C, as grandezas mensuradas tiveram diferenças numéricas da ordem de micrômetros. Assim evidenciou-se que, pelo modelo de Fraunhofer é possível determinação da das grandezas das fendas de difração.

Para as fendas múltiplas de mesmas dimensões, a regressão na figura \ref{fig:log} deu os resultados na tabela \ref{tab:regres} com $B \approx -1$. Com isso, o experimento também demostrou que a largura dos máximos primários de interferência depende inversalmente da quantidade de fendas, não apenas da separação delas. Novamente, mais um resultado que seria esperado, seguindo o modelo teórico.

Em relação às imagens observadas na figura \ref{fig:D}, as mesmas demonstraram que o padrão de difração observado no anteparo não só depende da largura, separação e quantidade de fendas, mas como também do formato e disposição das fendas. Assim, evidenciando que o padrões de difração e interferência observados podem ser muito diversos, dependendo fortemente das condições experimentais presentes, em especial, das características dimensionais das fendas.

