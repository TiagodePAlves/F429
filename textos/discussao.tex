\todo[inline,color=yellow]{Esta parte poderia ser integrada à seção de resultados, permitindo uma discussão mais direta. Por exemplo, suponha que você fez um ajuste de um modelo teórico aos dados experimentais. Este ajuste  pode ser apresentado diretamente com o gráfico da seção de "Resultados", porém sua discussão ficaria pendende. Portanto \textbf{(?)}}

\subsection{Passa-altas}

  Após a coleta de dados, foi possível perceber que o filtro construído seguia bem o modelo teórico. Inicialmente, é possível perceber que as frequências que deveriam ser filtradas (\SI{120}{\kilo\hertz}), com ganho de pelo menos \SI{-10}{\decibel}, tiveram atenuação \SI{-33(1)}{\decibel},como na tabela \ref{tab:transmitancia} indicando bom fator de filtragem. Além disso, a frequência que não deveria sofrer atenuação (\SI{8}{\kilo\hertz}), não teve grandes perdas. Como pode ser visto na tabela \ref{tab:transmitancia}, o ponto mais próximo de \SI{8}{\kilo\hertz}, \SI{8,3538(00008)}{\kilo\hertz}, teve ganho de \SI{-1,64(009)}{\decibel}, indicando boa passagem do sinal para a frequência em questão.\par
  Outro fator que mostra a eficácia do filtro, é o comportamento anômalo no diagrama de Bode na figura \ref{fig:boderc}. É possível perceber que, para alguns valores de frequência entre \SI{100}{\hertz} e \SI{10}{\hertz}, o osciloscópio não foi capaz de registrar com grande precisão as tensões de saída, causando flutuações nos valores de transmitância registrados. O grupo atribui tal fenômeno ao alto fator de atenuação nesta banda de frequência, cerca de \SI{-40}{\decibel}, fazendo com que o sinal não fosse intenso o suficiente para o osciloscópio utilizado. Também é possível perceber tal ocorrência para o gráfico de fase, em que flutuações muito abruptas ocorreram. Tal fenômeno ocorreu pois a fase do sinal estaria muito próxima de \SI{90}{\degree}, fazendo com que o osciloscópio registrasse o sinal com fase invertida, mostrando novamente o comportamento inesperado do osciloscópio devido à grande atenuação do sinal.
  \pagebreak
  
  \subsection{Passa-banda}