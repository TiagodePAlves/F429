\subsection{Circuitos RC}

Observando o gráfico \ref{fig:integrais}, é possível verificar, em azul, a operação de integração dos sinais originais em vermelho. Por definição ($V=Q/C$), logo, a carga no capacitor é diretamente proporcional à tensão no mesmo. dessa forma, os sinais integrados representam, na realidade a carga sobre o capacitor. Considerando este fenômeno, faz sentido que os sinais estejam em fase, dado que o carregamento no capacitor ocorre devido à tensão sobre o mesmo, sendo esta promovida pelo sinal de entrada.

Deve se considerar que o comportamento integrador ocorre quando altas frequências são aplicadas à filtros passa-baixas e que a frequência de corte do circuito integrador é $\omega_c=\SI{4,823(0004)}{\kilo\hertz}$, Dessa forma circuito, como a frequência testada para integração foi de \SI{50}{\kilo\hertz}, o resultado obtido foi coerente com o esperado pelo modelo teórico.

Já para o gráfico \ref{fig:derivadas}, a configuração de cores é análoga à anteriores, com vermelho representando os sinais de entrada e azul os sinais de saída, ou seja, os que indicam diferenciação. Os resultados obtidos mostraram-se condizentes com aqueles observados na literatura, sendo um indício do bom funcionamento do circuito empregado.

Seguindo um raciocínio análogo ao anterior, um circuito apresenta comportamento diferenciador quando é um filtro passa-altas submetido à baixas frequências. Como a frequência de corte é a mesma $\omega_c=\SI{4,823(0004)}{\kilo\hertz}$, porém a frequência empregada era baixa (\SI{500}{\hertz}, logo, pelo modelo teórico, o filtro deveria atuar como diferenciador, indicando mais uma vez que o circuito apresentava o comportamento desejado.

É importante mencionar que os mesmos componentes puderam ser utilizados tanto para a integração quanto para a diferenciação devido à frequência de corte. Como $\omega_c$ possuía um valor intermediário entre as frequências que seriam integradas e diferenciadas, quando os componentes eram configurados como filtro passa-baixas, as frequências altas eram filtradas pelo circuito e quando analisadas sofriam integração. Sendo que para a diferenciação, simplesmente o corria o análogo para um filtro passa-altas.

\subsection{Circuito RLC}

Do mesmo modo, no gráfico \ref{fig:amortecidos}, o sinal em azul mostra a resposta temporal de um circuito ressonante amortecido para um sinal transiente, em vermelho.
